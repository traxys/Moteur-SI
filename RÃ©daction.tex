\documentclass[12pt]{article}
\usepackage[utf8]{inputenc}
\usepackage[T1]{fontenc}
\usepackage[frenchb]{babel}
\usepackage{graphicx}
\usepackage{wrapfig}
%\DeclareGraphicsExtensions{.pdf,.png,.jpg,.eps}%
\begin{document}

\title{Projet de Sciences de L'ing�ieur : Moteur}
\author{QUentin Boyer,L�o Boudoin}

\maketitle

\tableofcontents

\section{Energies disponilble}
Pour induire une rotation differentes energies s'offrent � nous. On pourrait utiliser l'energie hydraulique , cependant les contraitnes de volumes et de praticit� rendent l'utilisattion de cette energie difficile , cela ne semble pas etre l'energie optimale.
De m�me l'energie electrique �tant difficilement utilisable avec seulement des mat�riaux de r�cup�ration il ne semble pas possible que cette energie soit utilis�e.
Aussi l'energie �olinne semble �tre trop volumineuse � metre en place et tres impratique , soit tres inadapt� � nos besoins.
L'energie musculaire est a exclure �tant donn� que le systeme se doit d'etre autonome , et pour des raisons �videntes l'energie nucl�aire ne peut �tre utilis�e.
Il nous reste donc l'energie thermique qui semple �tre suffisment simple , et adapt� � notre projet.
De cette �nergie , pour faire simplement une rotation il se d�gage deux options majeures:
\begin{itemize}
	\item Les moteurs stirlings
	\item Les moulins � air
\end{itemize}
Etant donn� que les moulins sont la solution semblant �tre la plus simple , elle donc selon le razoir d'Ockham la meilleure solution.

\section{Besoins et probl�matiques}

\section{R�alisation}

\end{document}
