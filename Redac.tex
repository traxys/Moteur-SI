\documentclass[a4paper,12pt]{article}
\usepackage[T1]{fontenc}
\usepackage[utf8]{inputenc}
\usepackage[frenchb]{babel}
\usepackage{lmodern}

\title{Projet de SI : Moteur Autonomne}
\author{Quentin Boyer}

\begin{document}

\maketitle
\tableofcontents

\section{Energies Disponibles}

\indent Pour induire une rotation de multiples énergies sont disponibles. Comme par exemple l'énergie hydraulique , qui est néanmoins peu pratique et est rendue inutilisable par des contraintes de volume. Pour ces mêmes raisons l'énergie éolienne ne semble pas pouvoir être utilisée , en plus d'être inconstante.\\[0.3cm]
\indent L'énergie électrique ne peut être exploitée car elle demander plus que des matériaux de récupération comme imposé par le sujet. L'énergie musuclaire n'est pas envisageable puisque le système ne serait pas autonome et pour des raisons évidentes l'énergie nucléaire n'est pas adapté.\\[0.3cm]
\indent Ainsi il semble que l'énergie thermique est le moyen le plus adapté dans le cas de notre projet.\\[0.05cm]

\pagebreak

\section{Solutions Techniques}
Dans le cadre de l'énergie thermique nous avons à notre portée deux solutions techniques, le première étant de créer un moteur stirling qui fonctionne grâce à la chaleur d'une bougie, mais dont la réalisation est assez complexe et précise. La solution technique la plus simple pour répondre au contraintes semble donc être un "moulin à bougie" dont la réalisation et le fonctionnement sont extrêmement simples et fiables. En effet il suffit donc de mettre une bougie sous une roue semblable à une roue à aube avec un guide pour que l'air chaud monte vers les pales , ce qui est nettement plus simple que et le moteur stirling et les moulins à axe vertical étant donné que le positionnement des pales n'influent pas dans le cas décrit ici.

  \section{Détails de la réalisation}
  Pour faire ce systeme il faut une source d'énergie et une sortie pour capter la rotation. Le bâti va aussi servir de guide à l'air chaud et de support a la roue pour une question de praticité et de simplicité. Nous allons ultiliser une roue à 8 pales pour être sûr que la roue soit suffisement entrainée par la chaleur sans être trop lourde , et que 8 pales est nombre aisement construcutible de façon symétrique.
\end{document}
