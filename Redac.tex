\documentclass[a4paper,12pt]{article}
\usepackage[T1]{fontenc}
\usepackage[utf8]{inputenc}
\usepackage{lmodern}

\title{Projet de SI : Moteur Autonomne}
\author{Quentin Boyer}

\begin{document}

\maketitle
\tableofcontents

\section{Energies Disponibles}

  Pour induire une rotation de multiples énergies sont disponibles. Comme par exemple l'energie hydraulique , qui est néanmoins peut pratique et est rendue inutilisable par des contraintes de volumes. Pour ces mêmes raisons l'energie éolienne ne semble pas pouvoir être utilisée , en plus d'être inconstante.\\[0.05cm]
  
  L'energie electrique ne peut etrê exploitée car elle demander plus que des matériaux de récupération comme imposé par le sujet.\\[0.05cm]
  
  L'energie musuclaire n'est pas envisageable puisque le systéme ne serait pas autonome et pour des raisons évidentes le nucléaire n'est pas adapté.
  Ainsi il semble que l'energie thermique est le moyen le plus adapté dans le cas de notre projet.\\[0.05cm]
\end{document}
