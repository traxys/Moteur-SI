\documentclass[a4paper,12pt]{article}
\usepackage[T1]{fontenc}
\usepackage[utf8]{inputenc}
\usepackage[frenchb]{babel}
\usepackage{lmodern}

\title{Projet de SI : Moteur Autonomne}
\author{Quentin Boyer}

\begin{document}

\maketitle
\tableofcontents

\section{Energies Disponibles}

\indent  Pour induire une rotation de multiples énergies sont disponibles. Comme par exemple l'energie hydraulique , qui est néanmoins peut pratique et est rendue inutilisable par des contraintes de volumes. Pour ces mêmes raisons l'energie éolienne ne semble pas pouvoir être utilisée , en plus d'être inconstante.\\[0.3cm]
\indent  L'energie electrique ne peut etrê exploitée car elle demander plus que des matériaux de récupération comme imposé par le sujet. L'energie musuclaire n'est pas envisageable puisque le systéme ne serait pas autonome et pour des raisons évidentes le nucléaire n'est pas adapté.\\[0.3cm]
\indent  Ainsi il semble que l'energie thermique est le moyen le plus adapté dans le cas de notre projet.\\[0.05cm]

\pagebreak

\section{Solutions Techniques}
  Dans le cadre de l'energie thermique nous avons à notre portée deux solutions techniques , premièrement nous pourrions faire un moteur stirling qui fonctionne grâce à la chaleur d'une bougie , mais dont la réalisation est assez complexe et précise. La solution technique la plus imple pour répondre au contraintes semble donc être un "moulin à bougie" dont la réalisation et le fonctionnement sont extremement simple et fiable. En effet il suffit donc de mettre une bougie sous une roue semblbla à une roue à aube avec un guide pour que l'air chaud monte vers les pales , ce qui est nettement plus simple que et le moteur stirling et les moulin à axe vertical étant donnée que le posionement des pales n'influent dans le cas décrit ici.
  
\section{Détails de la réalisation}
  Pour faire ce systeme il faut une source d'energie et une sortie pour capter la rotation. Le bâti va aussi servir de guide à l'air chaud et de support a la roue pour une question de praticité et de simplicité. Nous allons ultiliser une roue à 8 pales pour etre sur que la roue soit suffisement entrainé par la chaleur sans être trop lourde , et que 8 pâle est nombre aisement construcutible symétriquement.
\end{document}
